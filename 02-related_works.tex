\section{Related Works}

\subsection{Modularity in SAR Robots}
Modularity in SAR robotics has been explored extensively, offering benefits including adaptability to diverse terrains, robustness to failure, and cost-effective scalability. Modular self-reconfigurable systems -- composed of standardized building blocks with uniform docking and communication interfaces -- can dynamically alter their shape and capabilities in response to environmental demands. For instance, PolyBot \cite{PolyBot} demonstrates sequential reconfiguration from snake-like locomotion to legged configurations, enhancing mobility over rubble and uneven surfaces, a key requirement in urban rescue scenarios.

Similarly, self-reconfigurable modular robots have been proposed for generality and resilience, with reconfiguration allowing navigation in confined spaces and recovery from actuator failure. Such systems can eject or rearrange modules to continue operations even when components malfunction. These capabilities align well with SAR needs where mechanical damage or terrain variability is common. Comprehensive reviews in field robotics emphasize that modularity supports rapid adaptation to mission-specific requirements, with plug-and-play payloads enabling teams to swap in appropriate locomotion (e.g., wheels, tracks, legs) or sensor (e.g., thermal, gas) modules as needed \cite{ModularRobotic}.

Despite its promise, true deployment of self-reconfigurable systems in real-world SAR remains limited. Challenges include durable docking mechanisms, distributed control, and scalability to many modules. A 2002 survey \cite{wiredBots2002} noted that while modular robots offered versatility, practical issues such as module complexity and control reliability were significant obstacles \cite{}. Nonetheless, ongoing research -- such as morphogenetic and CPG-controlled modular designs \cite{CPGLocomotion} -- continues to push capabilities, targeting robustness and autonomy in SAR and field robotics applications.

\subsection{Acoustic-Based Victim Detection}
Acoustic sensing has emerged as a viable modality for victim detection in search and rescue (SAR) robotics, particularly in environments where vision systems may fail due to occlusions, smoke, or debris. Voice activity, breathing, or other human-generated sounds can be leveraged to identify the presence and location of victims trapped in disaster zones.

Recent advances in machine learning, particularly TinyML—machine learning models deployed on microcontrollers—have enabled real-time, low-power voice detection suitable for edge robotics. For instance, Banbury et al.~\cite{Banbury2021micronets} introduced MicroNets, a family of compact convolutional networks for keyword spotting and acoustic event detection on microcontrollers. Similarly, Warden et al.~\cite{Warden2018speech} demonstrated the feasibility of neural network-based speech recognition on constrained devices, which has been widely adopted for embedded voice interfaces.

On the embedded TinyML front, compact neural networks have been developed for audio classification on microcontrollers. For example, Barovic and Moin~\cite{Barovic2025tinyml} proposed a quantized 1D convolutional neural network capable of recognizing up to 23 speech commands and successfully deployed it on the Arduino Nano 33 BLE Sense using the Edge Impulse pipeline. The model achieved around 97\% accuracy in real-time inference with a footprint suitable for deployment on microcontrollers. Similarly, Viswanatha et al.~\cite{Viswanatha2022tinyml} demonstrated the implementation of TinyML-based voice keyword recognition—specifically “yes”, “no”, and “other”—on the same microcontroller platform using mel-spectrogram features and TensorFlow Lite Micro.

Although these works are not explicitly designed for search-and-rescue (SAR), they confirm the feasibility of real-time audio inference on resource-constrained embedded systems. Such findings support the potential application of embedded acoustic detection in autonomous SAR robotics.

In summary, the integration of acoustic-based detection using TinyML in robotics presents a lightweight and effective approach for real-time victim localization. Its advantages in terms of energy efficiency, low latency, and deployability on embedded hardware make it a promising direction for autonomous SAR systems.

\subsection{Indoor Localization in GPS-Denied Environments}
Amr et al. \cite{nagah2021enhanced} propose an enhanced RSSI-based indoor positioning technique that combines a log
distance path loss model with novel statistical corrections. In their method, thousands of BLE RSSI samples are collected at known distances and converted into distance estimates using the log-distance path model. From the distribution of these sample distances they compute beacon-specific correction parameters – effectively a distance correction factor ($\beta$) and an exponent ($\sigma$) for each beacon and distance setting. These correction parameters are stored in a lookup table and applied to adjust realtime RSSI-derived distances. Finally, an enhanced centroid localization algorithm uses the corrected distances to estimate the node’s coordinates. In experiments this approach yielded higher positioning stability and accuracy than standard centroid or trilateration methods, all while using only existing BLE beacons (no extra hardware). 

Zhang and Zhao \cite{WSNCentroid} introduce a complementary triangle-centroid algorithm for RSSI-based localization in wireless sensor networks. Their method first orders beacons by received RSSI and selects the three strongest; the average RSSI from each beacon is converted to a distance, defining three circles centered on the known beacon locations. The intersection points of these three circles (labeled M, L, N) are calculated by solving the circle equations. The unknown node’s position is then estimated as the centroid (geometric average) of the triangle M–L–N. In simulation this “triangle + centroid” approach significantly reduces localization error compared to ordinary trilateration. Zhang and Zhao show that taking the centroid of the intersecting area effectively mitigates RSSI fluctuations and yields more accurate fixes than simply averaging the anchors.

Although not explicitly related to SAR application, together these studies establish the foundation for our work. Both highlight that calibrating raw RSSI-to distance estimates (via per-beacon correction factors) and then using centroid-based coordinate computation can greatly improve low-cost indoor localization. Our ESP32-based system builds on these ideas by applying similar beacon-specific RSSI correction and a centroid localization scheme to achieve robust, hardware-efficient indoor positioning.