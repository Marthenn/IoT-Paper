\section{Introduction}
An effective and rapid response in the aftermath of natural or man-made disasters is a critical factor in the saving of human lives. Robotics has emerged as a vital tool for Search and Rescue (SAR) operations, capable of entering environments that are too hazardous for human first responders \cite{SARRobot}. However, a significant challenge persists: the vast diversity of disaster scenarios. The unique conditions of a collapsed building, a flooded area, or rugged wilderness terrain demand different robotic capabilities for mobility and sensing. This makes a single, monolithic robot design inefficient and impractical, highlighting the need for adaptable and versatile platforms.

To overcome this limitation, this paper introduces a modular multi-modal SAR robot designed on a "building block" principle. The core idea is to enable the rapid assembly and reconfiguration of the robot using specialized modules to match the specific demands of a rescue mission. This approach enhances operational flexibility, allowing teams to deploy a robot tailored for optimal performance in any given environment, a concept that is becoming increasingly important in the field of field robotics \cite{ModularRobotic}.

For victim detection, our system moves beyond simple visual searches, which are often ineffective in cluttered debris. We integrate a lightweight, power-efficient voice recognition module to detect human sounds, such as cries for help. This is achieved using Tiny Machine Learning (TinyML), where a compact neural network runs directly on a low-power microcontroller \cite{TinyML}. This on-board processing allows the robot to react to audio cues in real-time without depending on a stable, high-bandwidth connection to a remote operator, which is often unavailable in disaster zones.

Once a potential victim's sound is detected, pinpointing their location is the next crucial step. As GPS signals are unreliable or completely absent inside buildings or under rubble, our robot utilizes a WiFi Received Signal Strength Indicator (RSSI) localization system. By deploying two or more low-cost ESP32 modules as fixed beacons in the operational area, the robot uses the relative signal strength from each beacon to perform centroid based localization \cite{WSNCentroid}. This paper presents the design of this modular system and details the experimental validation of the TinyML-based sound recognition and the WiFi RSSI localization subsystems.
