\section{Related Works}

\subsection{Modularity in SAR Robots}
Modularity in SAR robotics has been explored extensively, offering benefits including adaptability to diverse terrains, robustness to failure, and cost-effective scalability. Modular self-reconfigurable systems -- composed of standardized building blocks with uniform docking and communication interfaces -- can dynamically alter their shape and capabilities in response to environmental demands. For instance, PolyBot \cite{PolyBot} demonstrates sequential reconfiguration from snake-like locomotion to legged configurations, enhancing mobility over rubble and uneven surfaces, a key requirement in urban rescue scenarios.

Similarly, self-reconfigurable modular robots have been proposed for generality and resilience, with reconfiguration allowing navigation in confined spaces and recovery from actuator failure. Such systems can eject or rearrange modules to continue operations even when components malfunction. These capabilities align well with SAR needs where mechanical damage or terrain variability is common. Comprehensive reviews in field robotics emphasize that modularity supports rapid adaptation to mission-specific requirements, with plug-and-play payloads enabling teams to swap in appropriate locomotion (e.g., wheels, tracks, legs) or sensor (e.g., thermal, gas) modules as needed \cite{ModularRobotic}.

Despite its promise, true deployment of self-reconfigurable systems in real-world SAR remains limited. Challenges include durable docking mechanisms, distributed control, and scalability to many modules. A 2002 survey \cite{wiredBots2002} noted that while modular robots offered versatility, practical issues such as module complexity and control reliability were significant obstacles \cite{}. Nonetheless, ongoing research -- such as morphogenetic and CPG-controlled modular designs \cite{CPGLocomotion} -- continues to push capabilities, targeting robustness and autonomy in SAR and field robotics applications.

\subsection{Acoustic-Based Victim Detection}
TODO: manuella

\subsection{Indoor Localization in GPS-Denied Environments}
TODO: obed \cite{WSNCentroid} ini sitasi ke paper yang dipake di ppt mu
