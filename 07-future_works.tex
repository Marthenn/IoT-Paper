\section{Future Works}
\begin{itemize}
  \item \textbf{Improve RSSI Robustness:} Enhance the localization model to be more resilient to environmental factors. Possible approaches include dynamic calibration of the RSSI-distance model, machine-learning-based correction factors, or fusing RSSI with other signal features to reduce multipath error.
  \item \textbf{Multi-sensor Fusion:} Incorporate additional sensors (e.g. IMU, cameras, LIDAR) to strengthen situational awareness. Fusing inertial data and vision (e.g. depth or optical flow) with RSSI could significantly improve localization and mapping accuracy. For example, inertial measurements would help track the robot’s motion, and a camera could visually verify detected sound sources.
  \item \textbf{Dynamic Docking and Reconfiguration} Develop better physical interfaces and software for automated module coupling. Improving the robot’s ability to dock new modules (e.g. swapping or adding sensors) and reconfigure its shape would extend adaptability. Future designs might use spring-loaded connectors or magnetic alignment and implement software for hot-swapping modules on the fly.
  \item \textbf{Autonomous Navigation and Decision-Making} Integrate autonomous mobility and higher-level planning into the SAR robot. Future work should enable the robot to navigate indoor spaces (e.g. via SLAM algorithms), make path-planning decisions, and autonomously approach detected sound sources. In addition, machine-learning could be used for decision-making, such as prioritizing search areas or coordinating multiple robot units in a rescue mission.
\end{itemize}
