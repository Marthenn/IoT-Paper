\documentclass[conference]{IEEEtran}
\IEEEoverridecommandlockouts
% The preceding line is only needed to identify funding in the first footnote. If that is unneeded, please comment it out.
\usepackage{cite}
\usepackage{amsmath,amssymb,amsfonts}
\usepackage{algorithmic}
\usepackage{graphicx}
\usepackage{textcomp}
\usepackage{booktabs}
\usepackage{xcolor}
\def\BibTeX{{\rm B\kern-.05em{\sc i\kern-.025em b}\kern-.08em
    T\kern-.1667em\lower.7ex\hbox{E}\kern-.125emX}}
\begin{document}

\title{Multi-modal SAR Robot}
% {\footnotesize \textsuperscript{*}Note: Sub-titles are not captured in Xplore and
% should not be used}
% \thanks{Identify applicable funding agency here. If none, delete this.}
% }

\author{
    \IEEEauthorblockN{
        Bintang Dwi Marthen\IEEEauthorrefmark{1}\IEEEauthorrefmark{3},
        Kenneth Ezekiel Suprantoni\IEEEauthorrefmark{1}\IEEEauthorrefmark{3},
        Manuella Ivana Uli Sianipar\IEEEauthorrefmark{1}\IEEEauthorrefmark{3},
        Reza Pahlevi Ubaidillah\IEEEauthorrefmark{1}\IEEEauthorrefmark{3}, \\
        Priya Qolbu Dhiya'an Amar\IEEEauthorrefmark{1},
        Alghifari Mahfudz Rumi\IEEEauthorrefmark{2},
        Nana Sutisna\IEEEauthorrefmark{1},
        Infall Syafalni\IEEEauthorrefmark{1}, and
        Trio Adiono\IEEEauthorrefmark{1}
    }
    \IEEEauthorblockA{
        \IEEEauthorrefmark{1}School of Electrical Engineering and Informatics, 
        Institut Teknologi Bandung, Bandung, Indonesia \\
        \{marthen.bintangdwi, ken35kiel, manuellaivanauli, rezapuobed, priyaqda\}@gmail.com, \{nsutisna, infall, tadiono\}@itb.ac.id
    }
    \IEEEauthorblockA{
        \IEEEauthorrefmark{2}Faculty of Mechanical and Aerospace Engineering, 
        Institut Teknologi Bandung, Bandung, Indonesia \\
        alghifari.mahfudz@gmail.com
    }
    \thanks{\IEEEauthorrefmark{3}These authors contributed equally to this work.}
}

\maketitle

\begin{abstract}
Search-and-rescue (SAR) operations demand robotic platforms that are adaptable to diverse and hazardous environments. This work presents a modular SAR robot that integrates on‑board TinyML-based human voice detection and RSSI‑based indoor localization using low‑cost ESP32 modules. The robot continuously processes audio via a compact convolutional neural network running on a microcontroller, achieving 97.79\% accuracy in distinguishing human vocalizations from background noise. Simultaneously, three stationary ESP32 beacons perform WiFi RSSI measurements and compute a weighted triangle‑centroid estimate of the robot’s position, yielding an average localization error of approximately 0.5m. Experimental results demonstrate the feasibility of embedding efficient intelligence in inexpensive hardware for real‑time victim detection and rough localization, highlighting the system’s potential to enhance SAR effectiveness in GPS‑denied, cluttered environments.
\end{abstract}

\begin{IEEEkeywords}
Search-and-rescue robotics, TinyML, RSSI localization, ESP32, human voice detection, modular robotics.
\end{IEEEkeywords}

\input{01-introduction}

\section{Related Works}

\subsection{Modularity in SAR Robots}
Modularity in SAR robotics has been explored extensively, offering benefits including adaptability to diverse terrains, robustness to failure, and cost-effective scalability. Modular self-reconfigurable systems -- composed of standardized building blocks with uniform docking and communication interfaces -- can dynamically alter their shape and capabilities in response to environmental demands. For instance, PolyBot \cite{PolyBot} demonstrates sequential reconfiguration from snake-like locomotion to legged configurations, enhancing mobility over rubble and uneven surfaces, a key requirement in urban rescue scenarios.

Similarly, self-reconfigurable modular robots have been proposed for generality and resilience, with reconfiguration allowing navigation in confined spaces and recovery from actuator failure. Such systems can eject or rearrange modules to continue operations even when components malfunction. These capabilities align well with SAR needs where mechanical damage or terrain variability is common. Comprehensive reviews in field robotics emphasize that modularity supports rapid adaptation to mission-specific requirements, with plug-and-play payloads enabling teams to swap in appropriate locomotion (e.g., wheels, tracks, legs) or sensor (e.g., thermal, gas) modules as needed \cite{ModularRobotic}.

Despite its promise, true deployment of self-reconfigurable systems in real-world SAR remains limited. Challenges include durable docking mechanisms, distributed control, and scalability to many modules. A 2002 survey \cite{wiredBots2002} noted that while modular robots offered versatility, practical issues such as module complexity and control reliability were significant obstacles \cite{}. Nonetheless, ongoing research -- such as morphogenetic and CPG-controlled modular designs \cite{CPGLocomotion} -- continues to push capabilities, targeting robustness and autonomy in SAR and field robotics applications.

\subsection{Acoustic-Based Victim Detection}
TODO: manuella

\subsection{Indoor Localization in GPS-Denied Environments}
TODO: obed \cite{WSNCentroid} ini sitasi ke paper yang dipake di ppt mu
aaa \cite{nagah2021enhanced}


\section{System Architecture}
The architecture of the proposed Search and Rescue robot is founded on the principle of mission-specific modularity. The primary objective of this design is to create a platform where both hardware and software can be adapted for different operational environments, such as collapsed buildings or open-field disaster sites, while maintaining reasonable costs. This approach allows a rescue team to augment a core robotic platform with specialized capabilities as needed.

Our methodology leverages a standard, commercially available robotic base—the Robotis TurtleBot—which provides robust and reliable locomotion. The TurtleBot chassis features a grid of pre-drilled mounting holes, which serves as the physical interface for our modular "building block" concept. New functional modules are physically attached to the chassis using a simple and direct screw-on system. Electronically, each module is wired directly to the main onboard processor, allowing for straightforward integration of custom components into the main system.

\subsection{Hardware Implementation}
The physical prototype is constructed upon a Robotis TurtleBot3 Burger, which serves as the core platform for locomotion and control. The primary onboard computer is a Raspberry Pi 4B with 4GB of RAM. The TurtleBot's Dynamixel servos are managed by a dedicated OpenCR board, which interfaces with the Raspberry Pi.

Our modular additions are integrated directly onto this base platform. The acoustic sensing capability is provided by a standard USB clip-on microphone. To enhance directed listening and improve the signal-to-noise ratio, the microphone is mounted at the focal point of a parabolic dish, which is attached to a servo motor for orientation control. For the localization task, the robot is equipped with a dedicated ESP32 microcontroller that functions as the onboard WiFi receiver and processor for RSSI measurements.

The localization infrastructure consists of three stationary beacons, each built from a standalone ESP32 module powered by AAA batteries. These beacons are positioned around the operational area to create the necessary reference grid. The complete hardware assembly is depicted in the CAD model shown in Fig.~\ref{fig:robotCAD}. To provide a clear overview, the key hardware components are summarized in Table~\ref{tab:hardware_components}.

\begin{figure}[!b]
  \centering
  \includegraphics[width=0.65\linewidth]{img/robotcad.png}
  \caption{CAD model of the assembled SAR robot, showing the TurtleBot3 Burger base, onboard electronics, and sensor attachments}
  \label{fig:robotCAD}
\end{figure}

\begin{table*}[t]
  \caption{Key Hardware Components}
  \label{tab:hardware_components}
  \centering
  \begin{tabular}{lll}
    \toprule
    \textbf{Component} & \textbf{Model/Part} & \textbf{Purpose} \\
    \midrule
    Core Platform & Robotis TurtleBot3 Burger & Mobility and base structure \\
    Central Processor & Raspberry Pi 4B (4GB) & Main control, audio processing \\
    Low-level Control & OpenCR Board & Interfacing with Dynamixel servos \\
    Onboard RSSI RX & ESP32 & Dedicated WiFi RSSI processing \\
    Microphone & USB Clip-on Microphone with Parabolic Dish & Capturing ambient audio \\
    Localization Beacons & ESP32 (x3) & Stationary WiFi signal sources \\
    Power Source & TurtleBot3 Battery Pack & System power \\
    \bottomrule
  \end{tabular}
\end{table*}

\begin{figure}[!b]
  \centering
  \includegraphics[width=0.65\linewidth]{img/layer.png}
  \caption{The layered IoT architecture of the system, showing the hierarchy from high-level application interfaces down to the physical perception and actuation components}
  \label{fig:layering}
\end{figure}

\subsection{Software and Data Flow}
The robot's architecture is organized into a four-tiered IoT stack, as illustrated in Fig.~\ref{fig:layering}. This layered approach separates high-level applications from the underlying hardware, promoting modularity. The stack consists of: the Application layer (user-facing dashboard and controller), the Middleware layer (handling data exchange via MQTT, Robot Operating System (ROS) 2, Micro-ROS, and WebSockets), the Communication layer (Wi-Fi and Serial USB), and the Perception \& Actuation Layer (physical sensors and actuators).

\begin{figure*}[!htbp]
  \centering
  \includegraphics[width=0.65\textwidth]{img/DFD.png}
  \caption{A detailed block diagram of the software components and data flow. The MQTT Broker on the Raspberry Pi acts as a central hub, routing data such as commands, position, and sound detection between the various hardware and software modules.}
  \label{fig:dfd}
\end{figure*}

The operational data flow between the primary software modules is managed using a hybrid approach that combines custom Python scripts, the MQTT protocol for lightweight messaging, and elements of the ROS for hardware interfacing. The specific interactions are detailed in Fig.~\ref{fig:dfd}. The audio classification task on the Raspberry Pi is implemented using the TensorFlow library.

The operational flow of the system is continuous and concurrent:
\begin{enumerate}
    \item \textbf{Acoustic Perception:} The main Python script on the Raspberry Pi continuously listens for audio captured by the parabolic microphone. It processes this audio stream in real-time using a TensorFlow model trained to recognize human voice patterns. The system constantly publishes the results—including the sound's orientation relative to the robot and a confidence level of human presence—to an MQTT topic.
    \item \textbf{Localization:} Concurrently, the dedicated onboard ESP32 continuously scans for WiFi signals from the three stationary beacons. Each beacon is configured to broadcast a unique and constant WiFi network name (SSID). The onboard ESP32 measures the RSSI from each beacon and uses a pre-calibrated model based on a least-squares mapping to calculate the robot's (x, y) coordinates relative to the known positions of the beacons.
    \item \textbf{Data Aggregation and Transmission:} The Raspberry Pi subscribes to the data streams from both the acoustic perception script and the ESP32 localization module. It aggregates this information—the robot's current location and the confidence/direction of any detected human voice—and transmits it to a remote operator's computer for monitoring and decision-making.
\end{enumerate}


\section{Methodology}
Lorem ipsum dolor sit amet, consectetur adipiscing elit. Vivamus gravida, elit ac porttitor ullamcorper, neque urna accumsan tellus, ut placerat sapien neque et est. Curabitur aliquet sem sed metus bibendum, eu tristique magna aliquam. Mauris quis nunc semper, suscipit est at, mollis neque. Pellentesque ut justo ac nisl semper tempus. Duis pretium enim at elit imperdiet, sagittis fringilla diam tempus. Sed fringilla finibus elit, vel rutrum tortor ullamcorper convallis. Etiam in turpis maximus, suscipit ipsum ut, feugiat mi. Sed ullamcorper diam quis ipsum scelerisque faucibus sed at ligula. Morbi lorem lorem, maximus ut mollis nec, tempor a libero. Sed vehicula fringilla metus vitae pharetra. Suspendisse tempor vitae orci id facilisis. Aliquam sed sem vel dui vehicula sagittis ac quis libero. Sed suscipit ipsum at pulvinar tempus. Cras ultrices accumsan libero lacinia vehicula.

\section{Results and Discussion}

\subsection{Voice Recognition Model Evaluation}
The proposed human voice detection model was evaluated on a held-out test set using multiple performance metrics, including accuracy, precision, recall, F1-score, and a confusion matrix. The test set comprised 3,622 audio clips, balanced across human and non-human classes.

The model achieved a final test accuracy of \textbf{97.79\%}, with a binary cross-entropy loss of 0.0909. The classification report is summarized in Table~\ref{tab:classification}.

\begin{table}[h]
\centering
\caption{Classification Performance on the Test Set}
\label{tab:classification}
\begin{tabular}{lcccc}
\toprule
\textbf{Class} & \textbf{Precision} & \textbf{Recall} & \textbf{F1-score} & \textbf{Support} \\
\midrule
Non-human (0) & 0.9723 & 0.9630 & 0.9676 & 1350 \\
Human (1)     & 0.9781 & 0.9837 & 0.9809 & 2272 \\
\midrule
\textbf{Accuracy} & \multicolumn{4}{c}{0.9760} \\
\textbf{Macro Avg} & 0.9752 & 0.9733 & 0.9743 & 3622 \\
\textbf{Weighted Avg} & 0.9760 & 0.9760 & 0.9760 & 3622 \\
\bottomrule
\end{tabular}
\end{table}

The confusion matrix in Table~\ref{tab:confusion} further illustrates the model’s performance:

\begin{table}[h]
\centering
\caption{Confusion Matrix}
\label{tab:confusion}
\begin{tabular}{ccc}
\toprule
 & \textbf{Predicted 0} & \textbf{Predicted 1} \\
\midrule
\textbf{Actual 0} & 1300 & 50 \\
\textbf{Actual 1} & 37 & 2235 \\
\bottomrule
\end{tabular}
\end{table}

The model shows strong recall for the human class (98.37\%), meaning it rarely misses vocal sounds in unseen test data—an important trait in SAR applications where false negatives (missed detections) are critical. The false positive rate is also low, indicating robustness against confusing non-human sounds.

These results confirm that a mel-spectrogram-based convolutional neural network, trained on real-world domestic sounds, can effectively distinguish human from non-human audio cues. The use of data augmentation techniques (e.g., time masking and additive noise) likely contributed to the model's generalization ability under variable acoustic conditions. Given its high accuracy and efficiency, this model is suitable for real-time deployment in embedded robotic platforms.

\subsection{Weighted Centroid Localization Evaluation}

\subsubsection{Regression Performance of RSSI–Distance Models}

Table ~\ref{tab:beacon_metrics} summarizes the test performance of the three per‑beacon neural models, evaluated both in log‑scale and converted to absolute error in centimeters.

\begin{table}[ht]
\centering
\caption{Test metrics for each beacon’s RSSI–distance regression model.}
\label{tab:beacon_metrics}
\begin{tabular}{lccc}
\toprule
\textbf{Beacon} & \textbf{MSE$ {\log}$} & \textbf{MAE ${\log}$} & \textbf{MAE (cm)} \\
\midrule
A & 0.00610 & 0.08499 & 45.16 \\
B & 0.00469 & 0.07139 & 47.08 \\
C & 0.01152 & 0.12102 & 63.36 \\
\bottomrule
\end{tabular}
\end{table}

Beacon-B achieves the lowest log‑scale MSE (0.0047) and MAE (0.0714), corresponding to an average distance error of about 47cm. Beacon-A is similar, with an MAE of 45cm. Beacon-C shows higher error (63cm), likely due to more variable signal propagation in its placement region. Overall, the per‑beacon regressors yield sub‑meter accuracy on the held‑out test set, demonstrating that the log‑distance model plus neural refinement can predict distance to within roughly 0.5m on average.

\subsubsection{Impact on Localization Accuracy}

When these distance estimates are used in the weighted triangle‑centroid localization, the variation in per‑beacon error propagates into the final position estimate.  In particular, Beacon-C’s larger MAE contributes more uncertainty to the intersection region.  However, by weighting each beacon inversely to its predicted radius, the algorithm down‑weights less reliable beacons, mitigating the impact of higher error sources.

\begin{itemize}
\item \textbf{Relative model quality.}  The discrepancy between Beacon-C and the other two beacons underscores the importance of environmental factors (e.g.\ obstructions or multipath) at each anchor.  In practice, anchors should be sited to minimize variability or supplemented with additional measurements (e.g.\ more samples or alternate channels).
\item \textbf{Localization robustness.}  Employing a weighted centroid ensures that beacons with consistently better predictions (A and B) dominate the final estimate, leading to more stable and accurate positions than a simple unweighted centroid.
\item \textbf{Short‑range suitability.}  Although sub‑meter ranging error is larger than specialized UWB systems, our approach uses only low‑cost ESP32 hardware and standard RSSI measurements.  This makes it well suited for applications where compact, battery‑powered nodes must be deployed without additional infrastructure.

\end{itemize}

Our experiments demonstrate that an RSSI‑based neural distance estimator, when paired with a weighted triangle‑centroid localization, yields reliable indoor positioning with average errors around 0.5m using only commodity ESP32 devices. However, these results are highly location‑dependent: when we deployed the same models in a different indoor environment (with distinct geometry and interference characteristics), the positioning accuracy degraded significantly, producing erratic and unreliable estimates. Thus, practical application will require per‑site calibration or adaptation to maintain sub‑meter performance.

\section{Conclusion}
We have presented a modular, low-cost SAR robot framework combining TinyML audio detection with RSSI ranging. The platform’s hardware (ESP32 microcontrollers and simple sensors) and software (embedded neural network) are inherently cheap and power-efficient, aligning with the “low cost, simplicity, and commonly-available elements” desirable in SAR systems. 

The TinyML speech classifier proved highly effective, achieving ~97.8\% human-voice detection accuracy on the embedded device. This result is in line with recent work on microcontroller-based speech recognition, confirming that compact neural models can run reliably on battery-powered SAR robots. The RSSI-based localization method successfully provided approximate indoor positioning (mean error $\approx$0.5 m) without additional hardware. In practice, however, its accuracy is coarse and environment-dependent. As reported in prior studies, RSSI localization can suffer error variations due to walls, multipath, and interference. Our experiments reflected this: while the average error was sub-meter in open areas, it increased in cluttered or reflective spaces. Thus, the RSSI approach proved useful for rough triangulation but has clear limitations in precision.

Overall, our multi-modal SAR robot demonstrates that embedding intelligence in inexpensive hardware is viable for rescue scenarios. The combination of voice-based victim detection and wireless localization adds capability beyond basic tele-operation. In future deployments, such a system could be used to listen for survivors and guide responders even in NLOS conditions. In sum, the results show that a minimalist, sensor-rich design can offer real-world value, enabling a cost-effective SAR robot.

\section{Future Works}
\begin{itemize}
  \item \textbf{Improve RSSI Robustness:} Enhance the localization model to be more resilient to environmental factors. Possible approaches include dynamic calibration of the RSSI-distance model, machine-learning-based correction factors, or fusing RSSI with other signal features to reduce multipath error.
  \item \textbf{Multi-sensor Fusion:} Incorporate additional sensors (e.g. IMU, cameras, LIDAR) to strengthen situational awareness. Fusing inertial data and vision (e.g. depth or optical flow) with RSSI could significantly improve localization and mapping accuracy. For example, inertial measurements would help track the robot’s motion, and a camera could visually verify detected sound sources.
  \item \textbf{Dynamic Docking and Reconfiguration} Develop better physical interfaces and software for automated module coupling. Improving the robot’s ability to dock new modules (e.g. swapping or adding sensors) and reconfigure its shape would extend adaptability. Future designs might use spring-loaded connectors or magnetic alignment and implement software for hot-swapping modules on the fly.
  \item \textbf{Autonomous Navigation and Decision-Making} Integrate autonomous mobility and higher-level planning into the SAR robot. Future work should enable the robot to navigate indoor spaces (e.g. via SLAM algorithms), make path-planning decisions, and autonomously approach detected sound sources. In addition, machine-learning could be used for decision-making, such as prioritizing search areas or coordinating multiple robot units in a rescue mission.
\end{itemize}


% \section*{Acknowledgment}

% The preferred spelling of the word ``acknowledgment'' in America is without 
% an ``e'' after the ``g''. Avoid the stilted expression ``one of us (R. B. 
% G.) thanks $\ldots$''. Instead, try ``R. B. G. thanks$\ldots$''. Put sponsor 
% acknowledgments in the unnumbered footnote on the first page.

% \section*{References}

% Please number citations consecutively within brackets \cite{b1}. The 
% sentence punctuation follows the bracket \cite{b2}. Refer simply to the reference 
% number, as in \cite{b3}---do not use ``Ref. \cite{b3}'' or ``reference \cite{b3}'' except at 
% the beginning of a sentence: ``Reference \cite{b3} was the first $\ldots$''

% Number footnotes separately in superscripts. Place the actual footnote at 
% the bottom of the column in which it was cited. Do not put footnotes in the 
% abstract or reference list. Use letters for table footnotes.

% Unless there are six authors or more give all authors' names; do not use 
% ``et al.''. Papers that have not been published, even if they have been 
% submitted for publication, should be cited as ``unpublished'' \cite{b4}. Papers 
% that have been accepted for publication should be cited as ``in press'' \cite{b5}. 
% Capitalize only the first word in a paper title, except for proper nouns and 
% element symbols.

% For papers published in translation journals, please give the English 
% citation first, followed by the original foreign-language citation \cite{b6}.

\bibliographystyle{IEEEtran}
\bibliography{references}

% \begin{thebibliography}{00}
% \bibitem{b1} G. Eason, B. Noble, and I. N. Sneddon, ``On certain integrals of Lipschitz-Hankel type involving products of Bessel functions,'' Phil. Trans. Roy. Soc. London, vol. A247, pp. 529--551, April 1955.
% \bibitem{b2} J. Clerk Maxwell, A Treatise on Electricity and Magnetism, 3rd ed., vol. 2. Oxford: Clarendon, 1892, pp.68--73.
% \bibitem{b3} I. S. Jacobs and C. P. Bean, ``Fine particles, thin films and exchange anisotropy,'' in Magnetism, vol. III, G. T. Rado and H. Suhl, Eds. New York: Academic, 1963, pp. 271--350.
% \bibitem{b4} K. Elissa, ``Title of paper if known,'' unpublished.
% \bibitem{b5} R. Nicole, ``Title of paper with only first word capitalized,'' J. Name Stand. Abbrev., in press.
% \bibitem{b6} Y. Yorozu, M. Hirano, K. Oka, and Y. Tagawa, ``Electron spectroscopy studies on magneto-optical media and plastic substrate interface,'' IEEE Transl. J. Magn. Japan, vol. 2, pp. 740--741, August 1987 [Digests 9th Annual Conf. Magnetics Japan, p. 301, 1982].
% \bibitem{b7} M. Young, The Technical Writer's Handbook. Mill Valley, CA: University Science, 1989.
% \end{thebibliography}
% \vspace{12pt}
% \color{red}
% IEEE conference templates contain guidance text for composing and formatting conference papers. Please ensure that all template text is removed from your conference paper prior to submission to the conference. Failure to remove the template text from your paper may result in your paper not being published.

\end{document}
